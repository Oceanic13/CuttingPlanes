\section{Mixed Integer Linear Program}

\begin{frame}{Mixed Integer Linear Program (MILP) and Relaxation}
%Some of the problem variables are restricted to be integer.
\small
\begin{columns}

	\column{0.5\textwidth}
	\begin{tcolorbox}[colback=white, title={MILP (standard form)}]
    \begin{align*}
    	\min_{x}\quad &c^\top x \\
    	\text{s.t.}\quad & x\in F_{MILP} \\
    	:= &\{x \:\vert\: Ax \leq b, x \in \mathbb{Z}_{\geq 0}^{n_1} \times \mathbb{R}_{\geq 0}^{n-n_1} \}
    \end{align*}
    \end{tcolorbox}

	\column{0.5\textwidth}    
	\begin{tcolorbox}[colback=white, title={LP Relaxation}]
    \begin{align*}
    	\min_{x}\quad &c^\top x \\
    	\text{s.t.}\quad &x \in P_{LP}\\
    	:= &\{x \:\vert\: Ax \leq b, x \in \mathbb{R}_{\geq 0}^{n}\}
    \end{align*}
    \end{tcolorbox}
\end{columns}
$c,x \in \mathbb{R}^n, A \in \mathbb{R}^{m \times n}, b \in \mathbb{R}^m$
\end{frame}
% Special cases: n1=0 -> LP (no integer constraints), n1=n -> ILP (all variables are integer)
% Mention Transformations: =, nonnegativity, slack variables, ... -> Transform to standard form

% Notation: P (for polyhedron) is the feasible region of the relaxation, F = P \cap (\mathbb{Z}^{n_1} \times \mathbb{R}^{n-n_1}) is the feasible region of the milp. Note that conx(F) is again a polyhedron but computing conv(F) is expensive.
\begin{frame}{Example}
\begin{columns}
	%TODO
	\column{0.5\textwidth}
	\begin{tcolorbox}[colback=white]
    \begin{align*}
    	\min_{x,y}\quad & -y \\
    	\text{s.t.}\quad & 3x + 2y &&\leq 6 \\
    	& -3x + y &&\leq 0 \\
    	& (x, y) &&\in \mathbb{Z}_{\geq 0} \times \mathbb{R}_{\geq 0}
    \end{align*}
	\end{tcolorbox}

	\column{0.5\textwidth}
	\begin{tcolorbox}[colback=white]
    \begin{align*}
    	\min_{x,y}\quad & -y \\
    	\text{s.t.}\quad & 3x + 2y &&\leq 6 \\
    	& -3x + y &&\leq 0 \\
    	& (x, y) &&\in \mathbb{R}_{\geq 0} \times \mathbb{R}_{\geq 0}
    \end{align*}
	\end{tcolorbox}
    
    \end{columns}
\end{frame}

\begin{frame}{Observations}
\begin{itemize}[<+->]
\item $c^\top x_{LP}^* \leq c^\top x_{MILP}^*$.

\item If $x_{LP}^* \in F_{MILP}$, then $c^\top x_{LP}^* = c^\top x_{MILP}^*$.
%meaning the relaxation is equivelent to the original MILP.

\item $x_{LP}^*$ can be found at a vertex of $P_{LP}$ (Simplex Algorithm).
% Assuming Bounded!
\end{itemize}
\end{frame}