\section{Branch \& Bound}

\begin{frame}{The Branch in Branch \& Bound}
Divide \& Conquer: Cut off the non-integer neighborhood of $\hat{x}_i \notin \mathbb{Z}$ and obtain two new relaxation problems, one with $x_i \leq \floor{\hat{x}_i}$ and one with $x_i \geq \ceil{\hat{x}_i}$. The resulting data structure is a binary tree of problems.

\begin{figure}[b]
\centering
\includesvg[]{imgs/branch.svg}
\end{figure}

\end{frame}

\begin{frame}{The Bound in Branch \& Bound}
\begin{itemize}
\item Initially, we only know that $-\infty < c^\top x^* < \infty$ (not very helpful).
\item Improve bounds:
\begin{itemize}
\item Lower Bound: For any LP-Relaxation, we have $c^\top \hat{x} \leq c^\top x^*$
\item Upper Bound: By definition, $c^\top x^* \leq c^\top x$ for any feasible $x \in F$
\end{itemize}
\item For an optimal solution $\hat{x}$ of a subproblem:
\begin{itemize}
\item if $c^\top \hat{x} \geq $ upper bound, prune tree (stop branching).
\item if $\hat{x}$ is feasible, update upper bound and prune.
\item if $\hat{x}$ is infeasible, update lower bound and branch.
\item stop if tree is completely pruned or upper bound $-$ lower bound $< \epsilon$.
\end{itemize}

\end{itemize}
\end{frame}

\begin{frame}{Example}
\only<1>{
\begin{figure}
\centering
\includesvg[height=0.8\textheight]{imgs/bb_idea_0.svg}
\end{figure}
}
\only<2>{
\begin{figure}
\centering
\includesvg[height=0.8\textheight]{imgs/bb_idea_1.svg}
\end{figure}
}
\only<3>{
\begin{figure}
\centering
\includesvg[height=0.8\textheight]{imgs/bb_idea_2.svg}
\end{figure}
}
\only<4>{
\begin{figure}
\centering
\includesvg[height=0.8\textheight]{imgs/bb_idea_3.svg}
\end{figure}
}
\only<5>{
\begin{figure}
\centering
\includesvg[height=0.8\textheight]{imgs/bb_idea_4.svg}
\end{figure}
}
\end{frame}

\begin{frame}{Subproblem and Branching Variable Selection Strategies}
\begin{itemize}
\item Subproblem selection:
\begin{itemize}
\item Depth-First: Descend quickly to obtain a good upper bound (obtain a feasible solution fast).
\item Best-First: Pick the active node with the current lower bound (obtain a good lower bound).
\end{itemize}

\item Branching variable selection:
\begin{itemize}
\item Most fractional: $i = \arg \max_{1 \leq i \leq n_1} \min (f_i, 1-f_i)$ ($f_i$ close to $\frac{1}{2}$).
\item Multiple variables at once.
\end{itemize}
\end{itemize}
\end{frame}

\begin{frame}{Cutting Planes + Branch \& Bound = Branch \& Cut }
\begin{itemize}
\item Cutting Planes or Branch \& Bound on their own are often inefficient in practice.
\item Combine the two to Branch \& Cut. This works like Branch \& Bound but with additionally adding cuts before branching. 
\item Often used in practice.
\end{itemize}
\end{frame}