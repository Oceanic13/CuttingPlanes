\section{Cutting Planes}

\begin{frame}{Cutting Planes}
Solve the problem relaxation. If integer constraints are violated, add additional inequalities to the problem.

\only<1>{
\centering
\begin{figure}[tb]
\includesvg[]{imgs/cp_idea_0.svg}
\end{figure}
}
\only<2>{
\centering
\begin{figure}[tb]
\includesvg[]{imgs/cp_idea_1.svg}
\end{figure}
}
\only<3>{
\centering
\begin{figure}[tb]
\includesvg[]{imgs/cp_idea_2.svg}
\end{figure}
}
\only<4>{
\centering
\begin{figure}[tb]
\includesvg[]{imgs/cp_idea_3.svg}
\end{figure}
}
\only<5>{
\centering
\begin{figure}[tb]
\includesvg[]{imgs/cp_idea_4.svg}
\end{figure}
}
\only<6>{
\centering
\begin{figure}[tb]
\includesvg[]{imgs/cp_idea_5.svg}
\end{figure}
}
\end{frame}

\begin{frame}{Valid Inequalities and Cuts}
\begin{itemize}[<+->]
	\item An inequality $a^\top x \leq r$ is \emph{valid} for a set $F$ if $a^\top x \leq r$ is satisfied for all $x \in F$.
	\begin{itemize}
		\item For example: $x \leq 2$ is a valid inequality for $\{x \in \mathbb{Z}_{\geq 0} \:\vert\: x \leq 2.718 \}$
	\end{itemize}
	\item A \emph{cutting plane} (or cut) w.r.t. $\hat{x} \in P \setminus F$ is any valid inequality $a^\top x \leq r$ for $F$ such that:
	\begin{equation*}
		a^\top \hat{x} > r
	\end{equation*}
\end{itemize}

%TODO
\end{frame}

\begin{frame}{Cutting Planes Algorithm}
     \begin{algorithmic}[1]
     \State LP $\gets$ Relaxation of the MILP
     \Repeat
    	\State $\hat{x}$ $\gets$ Optimal solution of the LP 
    	\If{$(\hat{x}_1,...,\hat{x}_{n_1}) \notin \mathbb{Z}^{n_1}$}
    		\State Add a cut w.r.t. $\hat{x}$ to the LP
    	\EndIf 
    \Until{$(\hat{x}_1,...,\hat{x}_{n_1}) \in \mathbb{Z}^{n_1}$}
    \State \textbf{return} $\hat{x}$
   \end{algorithmic}
\end{frame}

\begin{frame}[c]{Cutting Strategy}
\centering\large
	Question: How to generate ``good'' and useful cuts?
	\begin{itemize}[<+(1)->]
	\item Good: Cut away as much as possible (while staying feasible)
	\item Useful: Cut away the optimal solution of the relaxation
	\end{itemize}
\end{frame}

\begin{frame}{Convex Hull}
\begin{columns}
\column{0.5\textwidth}
\begin{itemize}
\item The relaxed solution $\hat{x}$ in $\text{conv}(F)$ also solves the MILP.
\item But computing the convex hull is infeasible (exponential). %TODO find reference
\end{itemize}

\column{0.5\textwidth}
\begin{figure}[p]
        \centering
        \includesvg[]{imgs/convex_hull.svg}
    \end{figure}
\end{columns}
\end{frame}

\begin{frame}{Integer Part and Fractional Part}
\begin{columns}

\column{0.5\textwidth}
Any real number $a \in \mathbb{R}$ can be expressed as
\begin{equation*}
a = \floor{a} + f_a
\end{equation*}
for some unique $\floor{a} \in \mathbb{Z}$ and $f_a \in [0,1)$.
\begin{itemize}
\item $\floor{a} = \max\{z \in \mathbb{Z} \:\vert\: z \leq a\}$ is the\newline
\emph{integer part} of $a$.
\item $f_a = a - \floor{a}$ is the\newline
\emph{fractional part} of $a$.
\end{itemize}


\column{0.5\textwidth}
\begin{itemize}[<+(1)->]
	\item $f_a = 0 \Leftrightarrow a=\floor{a} \Leftrightarrow a \in \mathbb{Z}$
	\item $\floor{-a} = -\ceil{a}$ where $\ceil{a} = \min\{z \in \mathbb{Z} \:\vert\: z \geq a\}$
	\item $a \in \mathbb{Z}$ and $a \leq b \Rightarrow a \leq \floor{b}$
	\item $a \in \mathbb{Z}$ and $a \geq b \Rightarrow a \geq \ceil{b}$
	\end{itemize}

\end{columns}
\end{frame}

\begin{frame}{Chvátal–Gomory Inequality for Integer Linear Programs}
Let $\sum_{j=1}^n a_{ij} x_j \leq b_i$ for an Integer Linear Program ($x \in \mathbb{Z}_{\geq 0}^n$).
Then the following inequalities are valid for any $\alpha \geq 0$:
\begin{enumerate}
	\item $\sum_{j=1}^n \alpha a_{ij} x_j \leq \alpha b_i$ \hfill $\alpha \geq 0$
	\item $\sum_{j=1}^n \floor{\alpha a_{ij}} x_j \leq \alpha b_i$ \hfill $x_j \geq 0$
	\item $\sum_{j=1}^n \floor{\alpha a_{ij}} x_j \leq \floor{ \alpha b_i}$ \hfill $x_j \in \mathbb{Z}$
\end{enumerate}
% Need xj >= 0. Otherwise, if aij,xj < 0 then possibly floor(aij)xj > aij xj

% Last Equation follows from x <= b for x integer => x <= floor(b) and the fact that if xj is int then also sum floor(...)x_j is integer
\end{frame}

\begin{frame}{Example}
Illustration %TODO
\end{frame}

\begin{frame}{Chvátal–Gomory Inequality}
If not all variables are integer, these inequalities are not valid.
Show that it does not work for mixed integer problem %TODO
\end{frame}

\begin{frame}{Basic Mixed Integer Rounding Inequalities I}
\only<1>{
Let $x \in \mathbb{Z}_{\geq 0}, y \in \mathbb{R}_{\geq 0}, b \in \mathbb{R}_{>0} \setminus \mathbb{Z}$. Then
\begin{equation}
x \leq \floor{b} \text{ is a valid inequality for } \{x+y \leq b\}
\end{equation}
and
\begin{equation}
x \geq \ceil{b} \text{ is a valid inequality for } \{-x+y \leq -b\}
\end{equation}
}

\only<2>{
Hello World
%TODO :  Illustration and math proof
}
\end{frame}

\begin{frame}{Basic Mixed Integer Rounding Inequalities II}
\only<1>{
Let $x \in \mathbb{Z}_{\geq 0}, y \in \mathbb{R}_{\geq 0}, b \in \mathbb{R}_{>0} \setminus \mathbb{Z}$. Then
\begin{equation}
x - \frac{1}{f_b-1} \leq \floor{b} \text{ is a valid inequality for } \{x-y \leq b\}
\end{equation}
and
\begin{equation}
x + \frac{1}{f_b} \geq \ceil{b} \text{ is a valid inequality for } \{-x-y \leq -b\}
\end{equation}
}

\only<2>{
%TODO :  Illustration and math proof
}

\end{frame}

\begin{frame}{General Mixed Integer Rounding Inequality}

\only<1>{
Let $F_{MIR} = \{(x,y) \in \mathbb{Z}_{\geq 0}^2 \times \mathbb{R}_{\geq 0} \:\vert\: a_1x_1+a_2x2-y\leq b \}$ where $a \in \mathbb{R}^2, b \in \mathbb{R} \setminus \mathbb{Z}$ and assume that $f_1 \leq f_b \leq f_2$. Then the inequality

\begin{equation*}
\floor{a_1}x_1 + \left( \floor{a_2}+ \frac{f_2 - f_b}{1 - f_b} \right) x_2 - \frac{1}{1-f_b}y \leq \floor{b}
\end{equation*}
is valid for $F_{MIR}$.
}

\only<2>{
%TODO Math Proof
}

\end{frame}

\begin{frame}{Simplex Algorithm}
Simplex finds $\hat{x} \in P \times \mathbb{R}_{\geq 0}^{N-n}$ and creates the optimal simplex tableau:
\begin{center}
\begin{minipage}{0.8\textwidth}
	\begin{tcolorbox}[colback=white, title={$i-$th row in the simplex tableau}]
    \begin{equation*}
    	x_{B_i} + \sum\limits_{\substack{j \in NB}} \bar{a}_{ij} x_j = \bar{b_i}
    \end{equation*}
    \end{tcolorbox}
\end{minipage}
\end{center}
\begin{columns}
\column{0.5\textwidth}
\begin{itemize}
\item $x_1,...,x_{n_1}$: Integer problem variables
\item $x_{n_1+1},...,x_n$: Real problem variables
\item $x_{n+1},...,x_N$: (Real) slack variables
\end{itemize}

\column{0.5\textwidth}
\begin{itemize}
\item $B=\{B_1,...,B_m\}$: Basic variables
\item $NB = \{1,...,N\} \setminus B$: Nonbasic variables ($\hat{x}_j=0$ for $j \in NB$)
\end{itemize}
\end{columns}
%Note: In some cases we also know that some slack variables must be integer
\end{frame}

\begin{frame}{Gomory Mixed Integer Cut}

\only<1>{
Let $N_1= NB \cap \{1,...,n_1\}, N_2 = NB \cap \{n_1+1,...,x_N\}$. Consider the $i-$th row in the optimal simplex tableau
\begin{equation*}
    x_{B_i} + \sum\limits_{\substack{j \in N_1}} \bar{a}_{ij} x_j + \sum\limits_{\substack{j \in N_2}} \bar{a}_{ij} x_j = \bar{b_i}
\end{equation*}
and assume $B_i \leq n_1$ but $\hat{x}_{B_i}=\bar{b_i} \notin \mathbb{Z}$. Then the \emph{Gomory Mixed Integer Cut}
\begin{tcolorbox}[colback=white]
\begin{equation*}
x_{B_i} + \sum\limits_{\substack{j \in N_1 \\ f_{ij} \leq f_i}} \floor{\bar{a}_{ij}} x_j + \sum\limits_{\substack{j \in N_1 \\ f_{ij} > f_i}} \left( \floor{\bar{a}_{ij}} + \frac{f_{ij}-f_i}{1-f_i} \right) x_j + \sum\limits_{\substack{j \in N_2 \\ \bar{a}_{ij} < 0}} \left( \frac{\bar{a}_{ij}}{1-f_i} \right) x_j \leq \floor{\bar{b_i}}
\end{equation*}
\end{tcolorbox}
% Remark: In literature often with four sums and ... >= f_b. One is obtained by subtracting the other from the initial row.
% Math Proof

is a valid inequality for $F$ that is not satisfied by $\hat{x}$.
}

\only<2>{
%TODO Proof
}

\end{frame}

\begin{frame}{Cutting Planes Algorithm}
\begin{itemize}
\item Let a MILP be given with feasible region $F = \{x \in \mathbb{Z}_{\geq 0}^{n_1} \times \mathbb{R}_{\geq 0}^{n-n_1} \:\vert\: Ax \leq b \}$ for some $A \in \mathbb{R}^{m \times n}, b \in \mathbb{R}^m$.
\item The relaxation is the LP obtained by removing the integer constraints, so its feasible region is the polyhedron $P = \{x \in \mathbb{R}_{\geq 0}^{n} \:\vert\: Ax \leq b \}$.
\item Repeat the following two steps until $\hat{x} \in F$:
\begin{enumerate}
\item Solve the LP using the Simplex Algorithm and obtain $\hat{x} \in P$
\item TODO %TODO
\end{enumerate}
\end{itemize}
\end{frame}

\begin{frame}{Project Demonstration}
%TODO
% Simplex Algorithm
% Mixed Integer Gomory Cuts
% Python Visualizer

%\begin{tcolorbox}[colback=white, title={MILP Standard Form}]
%    \begin{align*}
%    	\min_{x}\quad &-y \\
%    	\text{s.t.}\quad & 3x + 2y \leq 6 \\
%    	& -3x + 2y \leq 0 \\
%    	& x, y \in \mathbb{Z}_{\geq 0}
%    \end{align*}
%    \end{tcolorbox}
%    
%    %Also mention resubstitution of slack variables
\end{frame}

\begin{frame}{Selecting Cutting Planes}
\begin{itemize}[<+->]
\item Only adding an arbitrary, single cutting plane is very inefficient if the problem dimension is large.
\begin{itemize}
\item Heuristic to evaluate the efficiency of a cutting plane (e.g. euclidean distance to $\hat{x}$).
\item Add multiple cutting planes in each iteration.
\end{itemize}
\item Other cutting plane strategies exist.
%TODO Different Cutting Planes methods have been proposed/also depending on the specific problem
\end{itemize}
\end{frame}